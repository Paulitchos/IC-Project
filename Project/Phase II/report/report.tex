\documentclass[10pt]{article}

\PassOptionsToPackage{hidelinks}{hyperref}
\usepackage[utf8]{inputenc}
\usepackage{amsmath, calc, xcolor}
\usepackage{alphabeta} 
\usepackage[pdftex]{graphicx}
\usepackage[top=1in, bottom=1in, left=1in, right=1in]{geometry}
\usepackage[]{bookmark}
\linespread{1.06}
\setlength{\parskip}{8pt plus2pt minus2pt}

\widowpenalty 10000
\clubpenalty 10000

\newcommand{\eat}[1]{}
\newcommand{\HRule}{\rule{\linewidth}{0.5mm}}
\usepackage[official]{eurosym}
\usepackage{enumitem}
\setlist{nolistsep,noitemsep}
\usepackage[]{hyperref}
\usepackage{url}
\usepackage{cite}
\usepackage{lipsum}
\usepackage{indentfirst}
\usepackage{tikz}
\usetikzlibrary{arrows,decorations.pathmorphing,backgrounds,fit,positioning,shapes.symbols,chains}
\usepackage{xcolor,colortbl}
\usepackage{array}

\setlength{\parindent}{2em}
\renewcommand*\contentsname{Índice}
\renewcommand\refname{}

\begin{document}

%===========================================================
\begin{titlepage}
\begin{center}

% Top 
\includegraphics[width=0.55\textwidth]{img/logo-isec-transparente.png}~\\[2cm]


% Title
\HRule \\[0.4cm]
{ \LARGE 
  \textbf{Inteligência Computacional}\\[0.4cm]
}
\HRule \\[1.5cm]

% Docente
{ \large
  \textbf{Docente} \\[0.1cm]
  Inês Dominguês \\ Carlos Pereira \\[2.5cm]
}


% Author
{ \large
  \textbf{Alunos} \\[0.1cm]
  Paulo Henrique Figueira Pestana de Gouveia - a2020121705 \\[0.1cm]
  Nuno Alexandre Almeida Santos - a2019110035\\[0.1cm]
}

\vfill



% Bottom
{\large \today}
 
\end{center}
\end{titlepage}


\newpage



%===========================================================
\tableofcontents
\addtocontents{toc}{\protect\thispagestyle{empty}}
\newpage
\setcounter{page}{1}

%===========================================================
%===========================================================
\large
\section{Introdução}\label{sec:intro}

\section{Em que consiste a Computação Evolucionária?}\label{sec:apre-da-org}
A Computação Evolucionária compreende um conjunto de técnicas de busca e otimização inspiradas na 
evolução natural das espécies. Desta forma, cria-se uma população de indivíduos que vão reproduzir e 
competir pela sobrevivência. Os melhores sobrevivem e transferem suas características a novas gerações.

\subsection{Descrição do paradigma de computação evolucionária e possíveis aplicações no contexto de treino de uma rede neuronal}\label{sec:apre-da-org}

\section{Descrição da implementação dos algoritmos}\label{sec:Des-da-imp-dos-alg}

\section{Em que consiste a Inteligência Swarm}\label{sec:ev-da-org}
\subsection{Enquadrar a área designada de “inteligência Swarm” no domínio da Computação
            Evolucionária. Deve descrever sumariamente os principais paradigmas e algoritmos,
            em particular o algoritmo PSO.}\label{sec:apre-da-org}

\section{Funcionamento do algoritmo}\label{sec:an-da-info-fin-da-org}
\subsection{Descrever em detalhe o algoritmo selecionado e apresentar uma análise comparativamente com a versão base, o PSO \– 
            Particle Swarm Otimization. Quais as vantagens e desvantagens?}\label{sec:apre-da-org}

\section{Análsie de desempenho}\label{sec:an-da-info-fin-da-org}
\subsection{Aplicar e ilustrar ao algoritmo para otimização de duas funções “benchmark” \– a função esfera (para dimensão 2 e 3) e 
            função de Ackley (para dimensão 2 e 3). }\label{sec:apre-da-org}
\subsection{Comparar com a versão base do PSO e analisar a sensibilidade aos diferentes parâmetros do algoritmo, apresentando tabela 
            com resultados.}\label{sec:apre-da-org}

\section{Conclusões}\label{sec:an-da-info-fin-da-org}

\vspace{1cm}

\section{Referências}\label{sec:sup-inf-utl}
\bibliographystyle{ieeetr}
\bibliography{refs}


%===========================================================

%===========================================================

\pagebreak
\end{document} 
