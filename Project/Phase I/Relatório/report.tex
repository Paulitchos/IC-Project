\documentclass[10pt]{article}

\usepackage[utf8]{inputenc}
\usepackage{amsmath, calc, xcolor}
\usepackage{alphabeta} 
\usepackage[pdftex]{graphicx}
\usepackage[top=1in, bottom=1in, left=1in, right=1in]{geometry}
\usepackage[]{bookmark}
\linespread{1.06}
\setlength{\parskip}{8pt plus2pt minus2pt}

\widowpenalty 10000
\clubpenalty 10000

\newcommand{\eat}[1]{}
\newcommand{\HRule}{\rule{\linewidth}{0.5mm}}

\usepackage[official]{eurosym}
\usepackage{enumitem}
\setlist{nolistsep,noitemsep}
\usepackage[hidelinks]{hyperref}
\usepackage{url}
\usepackage{cite}
\usepackage{lipsum}
\usepackage{indentfirst}
\usepackage{tikz}
\usetikzlibrary{arrows,decorations.pathmorphing,backgrounds,fit,positioning,shapes.symbols,chains}
\usepackage{xcolor,colortbl}
\usepackage{array}


\setlength{\parindent}{2em}

\renewcommand*\contentsname{Índice}
\renewcommand\refname{}

\begin{document}

%===========================================================
\begin{titlepage}
\begin{center}

% Top 
\includegraphics[width=0.55\textwidth]{img/logo-isec-transparente.png}~\\[2cm]


% Title
\HRule \\[0.4cm]
{ \LARGE 
  \textbf{Inteligência Computacional}\\[0.4cm]
}
\HRule \\[1.5cm]

% Docente
{ \large
  \textbf{Docente} \\[0.1cm]
  Inês Dominguês \\ Carlos Pereira \\[2.5cm]
}


% Author
{ \large
  \textbf{Alunos} \\[0.1cm]
  Paulo Henrique Figueira Pestana de Gouveia - a2020121705 \\[0.1cm]
  Nuno Alexandre Almeida Santos - a2019110035\\[0.1cm]
}

\vfill



% Bottom
{\large \today}
 
\end{center}
\end{titlepage}


\newpage



%===========================================================
\tableofcontents
\addtocontents{toc}{\protect\thispagestyle{empty}}
\newpage
\setcounter{page}{1}

%===========================================================
%===========================================================
\large
\section{Introdução}\label{sec:intro}
Este trabalho foi realizado no âmbito da Unidade Curricular 
de Inteligência Computacional,tem por objetivo treinar uma rede 
neuronal capaz de estimar o valor da Bitcoin num determinado minuto.

\vspace{4cm}
\section{Descrição do caso de estudo e objetivos do problema}\label{sec:apre-da-org}
O Dataset escolhido foi Bitcoin Price USD, neste conjunto de dados os dados são gerados no intervalo 
de 1 minuto por uma API (Binance API) entre 1 de janeiro de 2021 a 12 de Maio de 2021.
Inclui várias colunas que mostram a mudança real no preço da Bitcoin também mostra o preço Open, High,
Low, Close da Bitcoin em minutos específicos. 
\begin{itemize}
    \item{Features}
    \begin{enumerate}
        \item Preço de abertura num minuto específico (Open Price of particular minute);
        \item Preço alto num minuto específico (High Price of particular minute);
        \item Preço baixo num minuto específico (Low Price of particular minute);
        \item Fechar Preço num minuto específico (Close Price of particular minute);
        \item Volume total num minuto específico (Total volume of particular minute);
        \item Volume de ativos de cotação (Quote asset volume);
        \item Número de negócios para determinado minuto (Number of trades for particular minute);
        \item Volume de ativos base de compra do tomador (Taker buy base asset volume);
        \item Volume de ativos de cotação de compra do tomador (Taker buy quote asset volume).
    \end{enumerate}
\end{itemize}
\begin{itemize}
    \item{Exemplos: 188318}
\end{itemize}   
Como o Horário de abertura (Open Time) e o Horário de fecho (Close Time) são sempre iguais, retirámos essas colunas das features.
\newpage
\section{Descrição da implementação dos algoritmos}\label{sec:ev-da-org}

Para treinar o nosso modelo optámos por usar o PyTorch que é uma biblioteca Python de aprendizagem máquina de código aberto.
A principal diferença entre PyTorch e TensorFlow é a escolha entre simplicidade e desempenho: o PyTorch é mais fácil de aprender
(especialmente para programadores Python), enquanto o TensorFlow tem uma curva de aprendizagem, mas tem um desempenho melhor e é mais utilizado na comunidade de desenvolvedores.
\vspace{1cm}
\\Os algoritmos de treino usádos foram: 
\vspace{1cm}
\begin{itemize}
\item MSE (Erro quadrado médio)
\vspace{1cm}
\begin{equation}
  \vspace{1cm}
  MSE : \sum_{i=1}^{D}(x_i-y_i)^2
\end{equation}
\item MAE (Erro absoluto médio)
\vspace{1cm}
\begin{equation}
  \vspace{1cm}
  MAE : \sum_{i=1}^{D}|x_i-y_i|
\end{equation}
\item RMSE (Raíz quadrada do erro médio)
\vspace{1cm}
\begin{equation}
  \vspace{1cm}
  RMSE : \sqrt{\frac{1}{n}\Sigma_{i=1}^{n}{\Big(\frac{d_i -f_i}{\sigma_i}\Big)^2}}
\end{equation}
\item RSQUARED (Coeficiente de determinação)
\vspace{1cm}
\begin{equation}
  \vspace{1cm}
\end{equation}
\newpage
\end{itemize}
Funções de ativação usadas:
\vspace{1cm}
\begin{itemize}
  \item Linear
  \vspace{1cm}
  \begin{equation}
    \vspace{1cm}
    Linear : xA^T + b
  \end{equation}
  \item AdaptiveLogSoftmaxWithLoss
  \vspace{1cm}
  \item RReLU
  \begin{equation}
    \vspace{1cm}
    RReLU(x) =
      \begin{cases}
            x, & \mbox{if } x \geq \mbox{0} \\
            ax, & \mbox{caso contrário}
      \end{cases}
  \end{equation}
  \item LogSigmoid
  \begin{equation}
    \vspace{1cm}
    LogSigmoid(x) =  \log(\frac{1}{1+\exp(-x)})
  \end{equation}
\end{itemize}
\vspace{1cm}
Treinámos 4 redes para testar valores:  
\vspace{1cm}
\begin{enumerate}
\item Rede MSE
  \begin{itemize}
    \item Nome - network MSE 2Lay 256 Linear 8 94e 05.tar
    \item Função de Treino - MSE
    \item Função de Ativação - Linear
    \item Nº de Neurónios - 256
    \item Nº de Camadas - 2
    \item Learning Rate - 3e-4
    \item Erro de teste - 8,94E-05
  \end{itemize}
  \vspace{2cm}
  \item Rede RMSE
  \begin{itemize}
    \item Nome - network RMSE 2Lay 256 Linear 0 00228.tar
    \item Função de Treino - RMSE
    \item Função de Ativação - Linear
    \item Nº de Neuronios - 256
    \item Nº de Camadas - 2
    \item Learning Rate - 3e-4
    \item Erro de teste - 0,0028
  \end{itemize}  
  \vspace{2cm}
  \item Rede MAE
  \begin{itemize} 
    \item Nome - network MAE 2Lay 256 Linear 0 00173 .tar
    \item Função de Treino - MAE
    \item Função de Ativação - Linear
    \item Nº de Neuronios - 256
    \item Nº de Camadas - 2
    \item Learning Rate - 3e-4
    \item Erro de teste - 0,00173
  \end{itemize}
  \vspace{2cm}
  \item Rede RSQUARED
  \begin{itemize}
    \item Nome - network R2 2Lay 256 Linear 0 062.tar
    \item Função de Treino - RSquared
    \item Função de Ativação - Linear
    \item Nº de Neurónios - 256
    \item Nº de Camadas - 2
    \item Learning Rate - 3e-4
    \item Erro de teste - 0,062   
  \end{itemize}
\end{enumerate}
  \newpage

\section{Análise de Resultados}\label{sec:ev-da-org}

\vspace{6cm}
\section{Conclusões}\label{sec:an-da-info-fin-da-org}

\vspace{3cm}

\newpage

\section{Referências}\label{sec:sup-inf-utl}
\cite{Binance:2021}

%===========================================================

%===========================================================

\pagebreak
\end{document} 
