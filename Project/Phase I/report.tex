\documentclass[10pt]{article}

\usepackage[utf8]{inputenc}

\usepackage{alphabeta} 
\usepackage[pdftex]{graphicx}
\usepackage[top=1in, bottom=1in, left=1in, right=1in]{geometry}

\linespread{1.06}
\setlength{\parskip}{8pt plus2pt minus2pt}

\widowpenalty 10000
\clubpenalty 10000

\newcommand{\eat}[1]{}
\newcommand{\HRule}{\rule{\linewidth}{0.5mm}}

\usepackage[official]{eurosym}
\usepackage{enumitem}
\setlist{nolistsep,noitemsep}
\usepackage[hidelinks]{hyperref}
\usepackage{url}
\usepackage{cite}
\usepackage{lipsum}
\usepackage{indentfirst}
\usepackage{tikz}
\usetikzlibrary{arrows,decorations.pathmorphing,backgrounds,fit,positioning,shapes.symbols,chains}
\usepackage{xcolor,colortbl}
\usepackage{array}


\setlength{\parindent}{2em}

\renewcommand*\contentsname{Índice}
\renewcommand\refname{}

\begin{document}

%===========================================================
\begin{titlepage}
\begin{center}

% Top 
\includegraphics[width=0.55\textwidth]{img/logo-isec-transparente.png}~\\[2cm]


% Title
\HRule \\[0.4cm]
{ \LARGE 
  \textbf{Inteligência Computacional}\\[0.4cm]
}
\HRule \\[1.5cm]

% Docente
{ \large
  \textbf{Docente} \\[0.1cm]
  Inês Dominguês \\ Carlos Pereira \\[2.5cm]
}


% Author
{ \large
  \textbf{Alunos} \\[0.1cm]
  Paulo Henrique Figueira Pestana de Gouveia - a2020121705 \\[0.1cm]
  Nuno Alexandre Almeida Santos - a2019110035\\[0.1cm]
}

\vfill



% Bottom
{\large \today}
 
\end{center}
\end{titlepage}


\newpage



%===========================================================
\tableofcontents
\addtocontents{toc}{\protect\thispagestyle{empty}}
\newpage
\setcounter{page}{1}

%===========================================================
%===========================================================
\large
\section{Introdução}\label{sec:intro}
Este trabalho foi realizado no âmbito da Unidade Curricular 
de Inteligência Computacional, 
tem por objetivo treinar uma rede neuronal capaz de estimar o valor da Bitcoin num determinado minuto.

\vspace{6cm}
\section{Descrição do caso de estudo e objetivos do problema}\label{sec:apre-da-org}
O Dataset escolhido foi Bitcoin Price USD, neste conjunto de dados os dados são gerados no intervalo 
de 1 minuto por uma API (Binance API) entre 1 de janeiro de 2021 a 12 de Maio de 2021.
Inclui várias colunas que mostram a mudança real no preço da Bitcoin também mostra o preço Open, High,
Low, Close da Bitcoin em minutos específicos. 
\begin{itemize}
    \item{Features}
    \begin{enumerate}
        \item Horário de abertura (Open Time);
        \item Preço de abertura num minuto específico (Open Price of particular minute);
        \item Preço alto num minuto específico (High Price of particular minute);
        \item Preço baixo num minuto específico (Low Price of particular minute);
        \item Fechar Preço num minuto específico (Close Price of particular minute);
        \item Volume total num minuto específico (Total volume of particular minute);
        \item Hora de fecho (Close Time);
        \item Volume de ativos de cotação (Quote asset volume);
        \item Número de negócios para determinado minuto (Number of trades for particular minute);
        \item Volume de ativos base de compra do tomador (Taker buy base asset volume);
        \item Volume de ativos de cotação de compra do tomador (Taker buy quote asset volume).
    \end{enumerate}
\end{itemize}
\begin{itemize}
    \item{Exemplos: 188318}
\end{itemize}   
Como o Horário de abertura (Open Time) e o Horário de fecho (Close Time) são sempre iguais, retirámos essas colunas das features.
\newpage
\section{Descrição da implementação dos algoritmos}\label{sec:ev-da-org}

\newpage

\section{Análise de Resultados}\label{sec:ev-da-org}

\vspace{6cm}
\section{Conclusões}\label{sec:an-da-info-fin-da-org}

\vspace{3cm}

\newpage

\section{Referências}\label{sec:sup-inf-utl}
\cite{Binance:2021}

%===========================================================

%===========================================================

\pagebreak
\end{document} 
